Wie unter Punkt \hyperref[covid]{\textit{Situation}} bereits erwähnt wurde für die Durchführung ein Arbeitsbereich geschaffen.
Die Entscheidung hierfür basierte auf der Einschätzung, dass die Gestaltung eines lokalen Arbeitsbereiches zeiteffizienter als der Weg in die Lehranstalt sei, da dieser durch Entfall von Präsenzunterricht nicht allfällig war.

Als Standort für diesen Arbeitsbereich boten sich insbesondere zwei Möglichkeiten an.

\begin{itemize}
    \item Eine gemeinschaftlich genutzte Werkstätte ca. zwei Kilometer vom Wohnort entfernt
    \item Eine Erweiterung des Arbeitsbereiches, welcher bereits in das eigene Zimmer integriert war.
\end{itemize}

Schlussendlich fiel die Entscheidung auf die zweite Option, vor allem aufgrund der Renovierung, welche die erste Möglichkeit frühestens zwei Wochen nach dem Treffen dieser Entscheidung zu einer validen Option gemacht hätte.

Die für die Gestaltung anfallenden Kosten konnten gering gehalten, sowie dadurch gerechtfertigt werden, dass dieser Arbeitsbereich auch nach Abschluss der Arbeit noch benutzt werden kann.

Insgesamt beliefen sich die Kosten für die Einrichtung des Bereiches auf etwa \textbf{300€}.
