\section{Fremderregte Gleichstrommaschine}
\label{fremd}

Fremderregte Gleichstrommotoren funktionieren nach dem gleichen Prinzip wie permanenterregte Motoren, mit dem Unterschied, dass das Statorfeld druch eine vom Rotor abhängige Spannungsquelle erzeugt wird.

Der größte Vorteil dieser Ansteurung ist die erhöhte Steuerbarkeit, da sowohl Stator- als auch Rotorspannung einzeln geregelt werden können.
Trotzdessen sind fremderregte Motoren nur noch selten anzutreffen.
Dies hat den Hintergrund, dass sie durch Drehstrommaschinen mit einem vorgeschalteten Frequenzumrichter ersetzt wurden, da diese eine ähnlich gut zu Steuern sind, allerdings simpler aufgebaut und somit billiger und leichter zu warten.

\cite{dcdewiki:208635995}