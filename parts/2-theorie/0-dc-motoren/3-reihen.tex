\section{Reihenschluss Gleichstrommaschine}
\label{reihen}

Auch beim Reihenschlussmotor wird das Statorfeld elektrisch erzeugt, jedoch in Reihe mit dem Rotor.
Man nennt diese Maschinen auch Hauptschlussmaschinen.

Bei ihnen ist das Drehmoment stark Drehzahlabhängig und sie können auch mit Wechselstrom betrieben werden, da eine Umpolung des Rotors auch zu einer Umpolung des Statorfeldes führt.
Dies hat zur Folge, dass die Drehrichtung gleich bleibt, jedoch muss bei Betrieb mit Wechselstrom der Stator durch ein Blechpaket ersetzt werden um Wirbelströme zu vermeiden.

\cite{dcdewiki:208635995}