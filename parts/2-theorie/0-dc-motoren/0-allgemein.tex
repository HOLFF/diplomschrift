Gleichstrommaschinen, auch Kommutator- oder Stromwendermaschinen genannt werden, wie der Name erkennen lässt, mit Gleichstrom betrieben und benötigen somit eine Möglichkeit den Stromfluss im Rotor zu wenden, da ansonsten keine Drehbewegung zu Stande kommen würde. 
Dieses Bauteil wird auch Stromwender oder Kommutator genannt, wodurch sich auch die anderen Namen erklären lassen.

Da also eine physische Verbindung zum Rotor nötig ist (meist Graphitbürsten), findet Abnutzung dieser statt, was ein Nachteil ist, da somit eine Fehlerstelle entsteht, sowie Wartung nötig ist.

Vorteile einer Kommutatormaschine sind die einfache Drehzahlregelung über die Spannung, sowie die Fähigkeit unter Last anzufahren.

Folgende sind die gängigsten Arten von Kommutatormaschinen.