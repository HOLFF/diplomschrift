\section{Permanenterregte Gleichstrommaschine}

Bei permanenterregten Gleichstrommotoren wird der Stator durch einen Permanentmagneten ersetzt.
Dies ist vor allem bei kleinen Anwendungen gebräuchlich, da trotz der technische Validität große Permanentmagneten zu teuer sind.

Gleich wie bei \hyperref[fremd]{\textit{fremderregten}} Gleichstrommaschinen haben auch permanenterregte Gleichstrommotoren hohe Einschaltströme.
Weiters kann das Magnetfeld nicht geschwächt werden, weshalb die Drehzahl weniger variiert werden kann.

Der Vorteil ist, das vor allem bei kleinen Motoren oft ein höherer Wirkungsgrad erreicht werden kann.

\cite{dcdewiki:208635995}