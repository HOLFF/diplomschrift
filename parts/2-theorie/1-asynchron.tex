\chapter{Dreh- und Wechselstrommaschinen}

Drehstrommaschinen unterscheiden sich von Gleichstrommaschinen, da diese wie der Name erraten lässt nicht von Gleichstrom, sondern von Drehstrom betrieben werden.
Hier ist einer der Vorteile, dass Frequenzumrichter verwendet werden können, um die Frequenz des verwendeten Stroms und somit auch die Drehzahl des Motors zu erhöhen.

\section{Synchronmaschine}

Bei Synchronmaschinen hat der Läufer ein eigenes Magnetfeld, welches entweder permanent ist, oder durch Fremderregung mittels Schleifringen erzeugt wird.
Man nennt diese Art Synchronmaschine, da der Läufer synchron zum Feld des Stators rotiert.
Bei Lastmoment entsteht Schlupf zwischen dem Läufer und dem Statorfeld, überschreitet dieses jedoch 90° bleibt der Läufer stehen.
\cite{sydewiki:204475526}

\section{Asynchronmaschine}

Die um ein Vielfaches gebräuchlichere Drehstrommaschine ist die Asynchronmaschine.
Dies ist durch ihre Vielzahl an Vorteilen zu erklären.

Hier kommt es durch die Rotation des Statorfeldes zu einer Flussänderung im Läufer, wodurch Strom induziert wird.
Je geringer der Schlupf (Unteschied zwischen der Rotation des Statorfeldes und dem Läufer) umso größer ist diese Flussänderung.
Dies bedeutet, dass der Läufer immer dem Statorfeld nachläuft, da bei gleicher Drehzahl keine Flussänderung, und somit keine Induktion stattfindet.

Durch diese Funktionsweise fällt die Nötigkeit von Schleifringen weg, was den Motor zur Verwendung in explosionsgefährdeten Bereichen qualifiziert.
Weiters macht ihn dies weniger wartungsintensiv, ebenso wie simpler zu produzieren.

Der größte Nachteil ist jedoch, dass diese Art Motor einen sehr hohen Einschaltstrom besitzt, da ohne Rotation des Läufers eine hohe Flussänderung herrscht, und somit viel Strom im Läufer induziert wird.
Aufgrunddessen müssen Asynchronmaschinen mit einem speziellen Anlassverfahren gestartet werden, um das Auslösen von Sicherungen sowie das Überlasten des Stromnetzes zu vermeiden.
\cite{asdewiki:206736532}