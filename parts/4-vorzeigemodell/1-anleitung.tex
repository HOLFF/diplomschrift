\section{Anleitung}

Der gegebene Aufbau kann durch spezifische Verkabelung entweder als \hyperref[neben]\textit{{Nebenschlussmaschine}}, oder als \hyperref[reihen]{\textit{Reihenschlussmaschine}} verwendet werden.

Bei Vorhandensein von zwei Spannungsquellen ist auch die Verwendung als \hyperref[fremd]{\textit{fremderregte}} Gleichstrommaschine möglich, dies wird hier jedoch nicht dokumentiert.

\subsection{Nebenschlussschaltung}

Um den Aufbau als Nebenschlussmaschine zu schalten müssen sowohl die Spulen, als auch die Schleifer parallel geschalten werden wie auf folgendem Bild.

[Bild Nebenschluss]

Um die Funktionsweise einer Nebenschlussmaschine zu beweisen können die beiden Verbindungen zu den Schleifern vertauscht werden, was die Drehrichtung umkehrt.

\subsection{Reihenschlussschaltung}

Um den Aufbau als Hauptschlussmaschine zu schalten müssen beide Schaltkreise in Reihe geschalten werden.
Das bedeutet, dass der positive Ausgang der Spannungsversorgung nur direkt mit einem der beiden Schleifer oder den Spulen verbunden werden darf.
Für den negativen Ausgang gilt das Gleiche.

[Bild Hauptschluss]

Hier kann bewiesen werden, dass bei Vertauschen der Verbindungen zu den Schleifern die Drehrichtung gleich bleibt.